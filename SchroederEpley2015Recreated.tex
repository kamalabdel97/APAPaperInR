\documentclass[man]{apa6}
\usepackage{lmodern}
\usepackage{amssymb,amsmath}
\usepackage{ifxetex,ifluatex}
\usepackage{fixltx2e} % provides \textsubscript
\ifnum 0\ifxetex 1\fi\ifluatex 1\fi=0 % if pdftex
  \usepackage[T1]{fontenc}
  \usepackage[utf8]{inputenc}
\else % if luatex or xelatex
  \ifxetex
    \usepackage{mathspec}
  \else
    \usepackage{fontspec}
  \fi
  \defaultfontfeatures{Ligatures=TeX,Scale=MatchLowercase}
\fi
% use upquote if available, for straight quotes in verbatim environments
\IfFileExists{upquote.sty}{\usepackage{upquote}}{}
% use microtype if available
\IfFileExists{microtype.sty}{%
\usepackage{microtype}
\UseMicrotypeSet[protrusion]{basicmath} % disable protrusion for tt fonts
}{}
\usepackage{hyperref}
\hypersetup{unicode=true,
            pdftitle={The Sound of Intellect: Speech Reveals a Thoughtful Mind, Increasing a Job Candidate's Appeal},
            pdfauthor={Kamal Abdelrahman},
            pdfkeywords={replication, R, employment, communication},
            pdfborder={0 0 0},
            breaklinks=true}
\urlstyle{same}  % don't use monospace font for urls
\usepackage{graphicx,grffile}
\makeatletter
\def\maxwidth{\ifdim\Gin@nat@width>\linewidth\linewidth\else\Gin@nat@width\fi}
\def\maxheight{\ifdim\Gin@nat@height>\textheight\textheight\else\Gin@nat@height\fi}
\makeatother
% Scale images if necessary, so that they will not overflow the page
% margins by default, and it is still possible to overwrite the defaults
% using explicit options in \includegraphics[width, height, ...]{}
\setkeys{Gin}{width=\maxwidth,height=\maxheight,keepaspectratio}
\IfFileExists{parskip.sty}{%
\usepackage{parskip}
}{% else
\setlength{\parindent}{0pt}
\setlength{\parskip}{6pt plus 2pt minus 1pt}
}
\setlength{\emergencystretch}{3em}  % prevent overfull lines
\providecommand{\tightlist}{%
  \setlength{\itemsep}{0pt}\setlength{\parskip}{0pt}}
\setcounter{secnumdepth}{0}
% Redefines (sub)paragraphs to behave more like sections
\ifx\paragraph\undefined\else
\let\oldparagraph\paragraph
\renewcommand{\paragraph}[1]{\oldparagraph{#1}\mbox{}}
\fi
\ifx\subparagraph\undefined\else
\let\oldsubparagraph\subparagraph
\renewcommand{\subparagraph}[1]{\oldsubparagraph{#1}\mbox{}}
\fi

%%% Use protect on footnotes to avoid problems with footnotes in titles
\let\rmarkdownfootnote\footnote%
\def\footnote{\protect\rmarkdownfootnote}


  \title{The Sound of Intellect: Speech Reveals a Thoughtful Mind, Increasing a
Job Candidate's Appeal}
    \author{Kamal Abdelrahman\textsuperscript{1}}
    \date{}
  
\shorttitle{Speech Increases a Job Candidate’s Appeal}
\affiliation{
\vspace{0.5cm}
\textsuperscript{1} City University of New York - Brooklyn College}
\keywords{replication, R, employment, communication\newline\indent Word count: X}
\usepackage{csquotes}
\usepackage{upgreek}
\captionsetup{font=singlespacing,justification=justified}

\usepackage{longtable}
\usepackage{lscape}
\usepackage{multirow}
\usepackage{tabularx}
\usepackage[flushleft]{threeparttable}
\usepackage{threeparttablex}

\newenvironment{lltable}{\begin{landscape}\begin{center}\begin{ThreePartTable}}{\end{ThreePartTable}\end{center}\end{landscape}}

\makeatletter
\newcommand\LastLTentrywidth{1em}
\newlength\longtablewidth
\setlength{\longtablewidth}{1in}
\newcommand{\getlongtablewidth}{\begingroup \ifcsname LT@\roman{LT@tables}\endcsname \global\longtablewidth=0pt \renewcommand{\LT@entry}[2]{\global\advance\longtablewidth by ##2\relax\gdef\LastLTentrywidth{##2}}\@nameuse{LT@\roman{LT@tables}} \fi \endgroup}


\DeclareDelayedFloatFlavor{ThreePartTable}{table}
\DeclareDelayedFloatFlavor{lltable}{table}
\DeclareDelayedFloatFlavor*{longtable}{table}
\makeatletter
\renewcommand{\efloat@iwrite}[1]{\immediate\expandafter\protected@write\csname efloat@post#1\endcsname{}}
\makeatother

\authornote{Kamal Abdelrahman is an undergraduate at the City
University of New York - Brooklyn College in Brooklyn, NY majoring in
psychology with a focus in statistical programming

Correspondence concerning this article should be addressed to Kamal
Abdelrahman, Postal address. E-mail:
\href{mailto:kamalabdel97@gmail.com}{\nolinkurl{kamalabdel97@gmail.com}}}

\abstract{
This study is an exact replication of Juliana Schroeder \& Nicholas
Epley's (2015) experiment of whether a potential job candidate is
percieved more intelligent through text or audio. 39 Forturne 500
company recruiters rated job candidates on their intellect, a composite
score of the candidate's intelligence, competence, and thoughtfulness.
They hypothesized that speech communicates intelligence better than
written words. This study recreated the analysis of regarding
presenation of pitches and their favorability. Analysis supported that
hypothesis.


}

\begin{document}
\maketitle

\section{Methods}\label{methods}

\emph{Participants}

In this study, there were 39 participants, all professional Fortune 500
recruiters. The average of the recruiters was \emph{M} = 30.85
(\emph{SD} = -6.24). 10.3\% of the participants are male and 76.9\% are
female.

\emph{Materials}

Two materials were used in this analysis. The first material was the
dataset used in Schroeder \& Epley's study from githhub. The data was
analyzed so that the t-test could be reproduced.

The second material was R Studio, the Integrated Development Environemnt
(IDE) for R. The IDE was used as a platform to analyze the dataset in R.

\emph{Procedure}

To recreate this analysis of the t-test, the following steps were taken.
First, the data was loaded into R from Github with the fread function
from the data.table library. Then, a difference\\
The data was retrieved from (website) and loaded into into a R via the
fread function available under the data.table library () The data

\section{Results}\label{results}

An independent samples t-test was conducted to examine the manipulation
effects of audio and written expressions of intelligence in potential
employees. Interviewees who expressed their intelligence through spokem
during their interviews were rated significantly better (\emph{M} =
6.43, \emph{SD} = 1.43) than interviewees who wrote out their responses
(\emph{M} = 4.39, \emph{SD} = 2.17) \(t(34.02) = -2.17\), \(p = .037\),
\(d_{s}\) = -1.13

\section{Discussion}\label{discussion}

Schroeder and Epley hypothesized that a person is a more appealing job
candidate if they communicated with their voice as opposed to with text.
Results supported this hypothesis, which stated that candidates who
communicated through audio were rated significantly more desirable than
canidates who did not. This analysis was consistent across all five
experiments that were conducted. This is consistent with other
literature that discusses

\newpage

\emph{Limitations} The overall findings of the study indicated that
candidates who communicated through audio as opposed to written text
were rated significantly more intelligent. Though, after reexamining the

\emph{Implications} The study explored the effects of various
communication methods in regards to a person's desirability. This has
potential for various avenues to be explored within the realm of
communication.

\section{References}\label{references}

\begingroup
\setlength{\parindent}{-0.5in} \setlength{\leftskip}{0.5in}

\hypertarget{refs}{}

\endgroup


\end{document}
